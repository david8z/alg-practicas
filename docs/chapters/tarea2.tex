\section{Tarea 2 - Distancias de edición con thresholds}
{\color{red}Realizado por David Alarcón, Víctor Callejas y José Mira}

Para optimizar el algoritmo se han implementado dos mejoras sobre los tres métodos de la tarea 1.
\subsection{Calcular alrededor de la diagonal}
Establecemos límites en el recorrido para que solamente se calculen aquellas partes del grafo de dependencias que tengan sentido para dicho umbral.
Es decir, zonas relativamente cercanas a la diagonal principal de la matriz. En nuestro caso, recorremos la matriz por filas.
        

\begin{lstlisting}[caption=Calculo de los umbrales]
lower_y = max(1, i - threshold)
upper_y = min(len(y), i + threshold)
\end{lstlisting}

\subsection{Detener ejecución sino promete}
Detenemos el algoritmo si, tras calcular una etapa en nuestro caso por filas, se puede asegurar que el coste superará el umbral.

\begin{lstlisting}[caption=Parar ejecución sino promete]
if min(M[i,:]) > threshold :
    return threshold + 1
\end{lstlisting}


\subsection{Testing}


Al igual que en la tarea 1, en el código se encuentra una test flag, por defecto activada, para que a la hora de importar este módulo se compruebe que el comportamiento de los algortimos es el adecuado, sino mostratá los test case en los que falla.

\newpage