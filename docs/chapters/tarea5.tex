\section{Tarea 5 - Estudio experimental}
En esta parte se pide realizar un estudio experimental de medidas de tiempo para determinar qué
versiones de las anteriores son las más eficientes para unos datos concretos
\subsection{Condideraciones}
Utilizamos dos diccionarios (castellano e inglés) ambos contruidos mediante el texto de la declaración de derechos humanos.

Sobre estos creamos múltiples diccionarios con las N([1000,5000,10000]) palabras más frecuentes. Para ello sobrecargamos el constructor de la clase Suggest de maneara que acepte un path al corpus y un set de palabras ya creado.

Sobre cada uno de estos se crean N consultas([10,100,500,1000,5000,10000]). Estas son palabras pertenecientes al deccionario sobre las que se realizan unas perturbaciones aleatorias.

\begin{lstlisting}[caption=Función para crear perturbaciones aleatorias]
def perturbar(word):

    word = list(word)

    n_ops = random.randint(0, min(len(word)-1,MAX_PERTURBACIONES))

    for _ in range(0,n_ops):

        op = random.randint(0, 3)

        if op == 0: # borrar
            idx = random.randint(0,len(word)-1)
            word = word[:idx] + word[(idx+1):]
        elif op == 1: # cambiar
            idx = random.randint(0,len(word)-1)
            char = chr(random.randint(ord('a'),ord('z')+1))
            word[idx] = char
        elif op == 2: # trasposicion
            idx = random.randint(0,len(word)-2)
            tmp = word[idx]
            word[idx] = word[idx+1]
            word[idx+1] = tmp

    return "".join(word)
\end{lstlisting}

Estas consultas se realizan con todos los algoritmos desarrollados en las tareas previas.

Utilizamos el mismo diccionario y consultas con todos los algoritmos (datos apareados).

Repetimos 3 veces la medición de tiempo de cada algoritmo para cada talla de diccionario y consultas para luego poder explorar los resultados y ver que son consistentes y válidos.
\begin{lstlisting}[caption=medición de tiempos]
    start = time.time()
                        
    for consulta in tqdm(consultas,total=len(consultas),leave=False,desc='Consultas: ',position=4):
        _ = iss.suggest(consulta,distance=alg,threshold=STATIC_THRESHOLD)

    end = time.time()
    elapsed = end - start
\end{lstlisting}

\subsection{Resultados}



\newpage 