\section{Tarea 3 - Metodo Suggest}
{\color{red}Realizado por David Alarcón, Víctor Callejas y José Mira}

Este método, se compara cada palabra con todas las del diccionario, ello conlleva un gran coste temporal.
Por ello se pide implementar alguna cota optimista que nos permita saber cuando una distancia será
mayor al umbral establecido.

\subsection{Límites teóricos superior e inferior}
La distancia de Levenshtein tiene varios límites superior e inferior simples. Éstas incluyen:
\begin{enumerate}
    \item Es al menos la diferencia de tamaños de las dos cuerdas.
    \item Es como máximo la longitud de la cuerda más larga.
    \item Es cero si y solo si las cadenas son iguales.
    \item Si las cuerdas son del mismo tamaño, la distancia de Hamming es un límite superior en la distancia de Levenshtein. La distancia de Hamming es el número de posiciones en las que los símbolos correspondientes en las dos cadenas son diferentes.
    \item La distancia entre dos cadenas Levenshtein no es mayor que la suma de sus distancias levenshtein de una tercera cadena (desigualdad triangular).
\end{enumerate}


\subsection{Implementación}

Nosotros hemos decidido implementar las condiciones 1 y 3.
Las condiciones 2 y 4, no las hemos implementado ya que estás no son cotas optimistas.
La condición 5 tampoco, ya que conlleva un sobrecoste en el cálculo de la distancia con la tercera palabra.

\begin{lstlisting}[caption=Condición 1]
if threshold is not None:
    lower = abs(length - len(word))
    if lower > threshold:
        continue
\end{lstlisting}

\begin{lstlisting}[caption=Condición 3]
if term == word:
    results[word] = 0
\end{lstlisting}

\subsection{Testing}
{\color{red}Realizado por David Alarcón}

Hemos modificado el archivo de leer{\_}resultados.py lo que nos permite comparar los resultados proporcionados por el profesor con los resultados obtenidos por nuestro suggester.
\newpage
Para ejecutar los test desde el directorio tarea 3 ejecutamos:

\begin{lstlisting}[language=bash]
    $ python leer_resultados.py
\end{lstlisting}

Además se permite especificar por parametros la distancia a usar \emph{levenshtein}, \emph{restricted} o \emph{intermediate} y si queremos que sea no verbos \emph{-nv}.

\newpage 