\section{Tarea 1 - Distancias de edición}
Implementaión de distancias de edición entre cadenas de forma iterativa y mediante programación dinámica.

\subsection{Distancia de Levensthtein}
{\color{red}Realizado por José Mira}

Considerando las operaciones de inserción, borrado y sustitución con coste = 1.

\begin{lstlisting}[caption=Algoritmo distancia de levenshtein]
    def dp_levenshtein_backwards(x, y):
        mat = matriz(x, y)
        res = np.zeros(shape=(len(x)+1,len(y)+1))
        for i in range(0,len(x)+1):
            for j in range(0,len(y)+1):
                    if i==0 or j==0:
                        res[i,j] = res[i,j] + i + j
                    else:
                        res[i,j] = min(
                            mat[i-1,j-1] + res[i-1,j-1],
                            1 + res[i-1,j],
                            1 + res[i,j-1]
                        )

        return res[len(x),len(y)]
\end{lstlisting}


\subsection{Distancia de Damerau-Levensthtein restringida}
{\color{red}Realizado por David Alarcón}

Se añade la operación de trasposición. En esta versión una vez intercambiados dos
símbolos, éstos no se pueden utilizar en otras operaciones de edición.

\begin{lstlisting}[label={list:first},caption=Sample Python code -- Damerau-Levensthtein restringido]
    def dp_restricted_damerau_backwards(x, y):
        # Simula el infinito
        INF = len(x) + len(y)

        mat = matriz(x, y)
        res = np.zeros(shape=(len(x)+1,len(y)+1))

        for i in range(0,len(x)+1):
            for j in range(0,len(y)+1):
                    if i==0 or j==0:
                        res[i,j] = res[i,j] + i + j
                    elif i == 1 or j == 1:
                        res[i,j] = min(
                            mat[i-1,j-1] + res[i-1,j-1],
                            1 + res[i-1,j],
                            1 + res[i,j-1],
                        )
                    else:
                        res[i,j] = min(
                            mat[i-1,j-1] + res[i-1,j-1],
                            1 + res[i-1,j],
                            1 + res[i,j-1],
                            1 + res[i-2,j-2] + (mat[i-2,j-1] + mat[i-1,j-2]) * INF
                        )
        return res[len(x),len(y)]
\end{lstlisting}

\vfill
\subsection{Distancia de Damerau-Levensthtein intermedia}
{\color{red}Realizado por Víctor Callejas}

Considerando las operaciones de trasposición cuando:
\begin{equation}
    |u| + |v| 	\leqslant cte \Leftarrow cte = 1
\end{equation}
\begin{lstlisting}[label={list:first},caption=Damerau-Levensthtein intermedio]
    def dp_intermediate_damerau_backwards(x, y):
        init_matriz(x, y)

        for i in range(1, len(x) + 1):
            for j in range(1, len(y) + 1):

                if x[i - 1] == y[j - 1]:
                    initActual = min(M[i-1, j] + 1, M[i, j-1] + 1, M[i-1][j-1])
                else:
                    initActual = min(M[i-1, j] + 1, M[i, j-1] + 1, M[i-1][j-1] + 1)

                if j > 1 and i > 1 and x[i - 2] == y[j - 1] and x[i - 1] == y[j - 2]:
                    M[i,j] = min(initActual, M[i-2][j-2] + 1)
                elif j > 2 and i > 1 and x[i-2] == y[j-1] and x[i-1] == y[j-3]:
                    M[i,j] = min(initActual, M[i-2][j-3] + 2)
                elif i > 2 and j > 1 and x[i - 3] == y[j-1] and x[i-1] == y[j-2]:
                    M[i,j] = min(initActual, M[i-3][j-2] + 2)
                else:
                    M[i,j] = initActual

        return M[len(x), len(y)]
\end{lstlisting}

\subsection{Testing}
En el código se encuentra una test flag, por defecto activada, para que a la hora de importar este módulo se compruebe que el comportamiento de los algortimos es el adecuado, mediante los test proporcionados en las diapositivas, sino mostratá los test case en los que falla.

\newpage